\chapter{PENDAHULUAN}
\section{Latar Belakang}
    Latar belakang merupakan subbab dari BAB I Pendahuluan. Bagian ini menguraikan berbagai hal yang menjadi dasar dilakukannya penelitian ini. Latar belakang umumnya menggunakan konsep segitiga terbalik, yaitu dimulai dari uraian umum lalu mengerucut kepada hal yang khusus. Konsep sebaliknya juga bisa digunakan, yaitu dimulai dari hal yang khusus kemudian sampai kepada hal yang lebih umum.
    
    Latar belakang dapat memuat penelitian terdahulu seperti merujuk pada buku \citep{altac2025}, artikel ilmiah \citep{ramli2024}, atau tesis \cite{vinarky2026}. \lipsum[1]
    
\section{Rumusan Masalah}
    Berdasarkan beberapa hal yang telah dipaparkan pada latar belakang, diperoleh beberapa rumusan masalah berikut.
    \begin{enumerate}
        \item Rumusan masalah pertama
        \item Rumusan masalah kedua
        \item Rumusan masalah ketiga
    \end{enumerate}
    
\section{Maksud dan Tujuan}
    Tujuan dari penelitian ini yaitu
    \begin{enumerate}
        \item menjawab rumusan masalah pertama,
        \item menyelesaikan rumusan masalah kedua,
        \item menyelesaikan rumusan masalah ketiga.
    \end{enumerate}
    
\section{Manfaat Penelitian}
    Penelitian ini dapat menambah wawasan penulis dan pembaca mengenai topik penelitian ini. Selain itu penelitian ini dapat menjadi referensi tambahan bagi peneliti berikutnya yang ingin membahas topik serupa.