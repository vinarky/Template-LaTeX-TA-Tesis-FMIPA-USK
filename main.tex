% =============================================================================
% KONFIGURASI DOKUMEN LATEX - TEMPLATE TATESIS (FMIPA USK)
% =============================================================================

% --- Pengaturan Jenis Dokumen ---
\documentclass{TATesis} % Menggunakan class khusus TATesis sebagai basis dokumen

% --- Pemanggilan Paket Tambahan ---
\usepackage{lipsum}     % Digunakan untuk menghasilkan teks dummy (placeholder)

% --- Identitas dan Input Data Penelitian ---
\jenis{1}               % Kategori dokumen: 1=Prop. TA, 2=Prop. Tesis, 3=TA, 4=Tesis
\judulID{Judul Indonesia} % Judul utama dalam Bahasa Indonesia
\judulEN{English Title}   % Terjemahan judul resmi dalam Bahasa Inggris
\nama{Nama Lengkap}    % Nama lengkap penulis sesuai identitas resmi
\npm{240XXXXXXX}        % Nomor Pokok Mahasiswa
\prodi{Matematika}      % Nama Program Studi (tulis tanpa awalan Sarjana/Magister)
\departemen{Matematika} % Nama Departemen atau Fakultas
\bulan{Januari, 2026}   % Bulan dan tahun terkini untuk penanggalan dokumen
\pembimbingSatu{Dr. Satu}{19XXXXXXXX} % Nama & NIP Dosen Pembimbing Utama
\pembimbingDua{Dr. Dua}{19XXXXXXXX}   % Nama & NIP Dosen Pembimbing Pendamping
\koordinator{Dr. Koord}{19XXXXXXXX}   % Nama & NIP Koordinator Program Studi

% --- Data Tambahan (Opsional untuk Dokumen Proposal) ---
\dekan{Prof. Dr. Dekan}{19XXXXXXXX}    % Nama & NIP Dekan untuk pengesahan akhir
% Bagian di bawah ini dilengkapi HANYA setelah melaksanakan sidang:
\hariSidang{hari Rabu, 21 Januari 2026} % Hari dan tanggal pelaksanaan sidang
\hariSah{hari Kamis, 22 Januari 2026}   % Hari dan tanggal pengesahan dokumen revisi

% =============================================================================
% MULAI ISI DOKUMEN
% =============================================================================
\begin{document}

% --- Bagian Depan (Front Matter) ---
\halamanJudul           % Menghasilkan halaman sampul depan
\halamanPengesahan      % Menghasilkan lembar pengesahan pembimbing/penguji

% Pernyataan & Abstrak di bawah ini umumnya hanya untuk laporan akhir (TA/Tesis).
% Berikan tanda komentar (%) di awal baris jika sedang membuat PROPOSAL.
\bebasPlagiasi          % Menghasilkan surat pernyataan orisinalitas karya
\abstrakID{kunci1}{isi} % Parameter 1: Kata Kunci | Parameter 2: Teks Abstrak (Ind)
\abstrakEN{key1}{text}  % Parameter 1: Keywords    | Parameter 2: Abstract Text (Eng)

\kataPengantar{
    \begin{enumerate}
        \item Fulan      % Ucapan terima kasih kepada pihak ke-1
        \item Fulan      % Ucapan terima kasih kepada pihak ke-2
        \item dst        % Dan pihak-pihak lainnya
    \end{enumerate}
}

% --- Daftar Tabel dan Gambar Otomatis ---
\daftarIsi              % Menghasilkan Daftar Isi
\daftarGambar           % Menghasilkan Daftar Gambar
\daftarTabel            % Menghasilkan Daftar Tabel
\daftarLampiran         % Menghasilkan Daftar Lampiran

% --- Pengaturan Penomoran Halaman Inti ---
\clearpage              % Berpindah ke halaman baru
\fancyhf{}              % Reset header dan footer agar kosong
\pagenumbering{arabic}  % Mulai penomoran halaman dengan angka (1, 2, 3...)
\fancyfoot[R]{\thepage} % Atur posisi nomor halaman di sisi kanan bawah

% --- Bagian Utama (Body Matter) ---
% Menggunakan perintah \input untuk memanggil file .tex terpisah
\chapter{PENDAHULUAN}
\section{Latar Belakang}
    Latar belakang merupakan subbab dari BAB I Pendahuluan. Bagian ini menguraikan berbagai hal yang menjadi dasar dilakukannya penelitian ini. Latar belakang umumnya menggunakan konsep segitiga terbalik, yaitu dimulai dari uraian umum lalu mengerucut kepada hal yang khusus. Konsep sebaliknya juga bisa digunakan, yaitu dimulai dari hal yang khusus kemudian sampai kepada hal yang lebih umum.
    
    Latar belakang dapat memuat penelitian terdahulu seperti merujuk pada buku \citep{altac2025}, artikel ilmiah \citep{ramli2024}, atau tesis \cite{vinarky2026}. \lipsum[1]
    
\section{Rumusan Masalah}
    Berdasarkan beberapa hal yang telah dipaparkan pada latar belakang, diperoleh beberapa rumusan masalah berikut.
    \begin{enumerate}
        \item Rumusan masalah pertama
        \item Rumusan masalah kedua
        \item Rumusan masalah ketiga
    \end{enumerate}
    
\section{Maksud dan Tujuan}
    Tujuan dari penelitian ini yaitu
    \begin{enumerate}
        \item menjawab rumusan masalah pertama,
        \item menyelesaikan rumusan masalah kedua,
        \item menyelesaikan rumusan masalah ketiga.
    \end{enumerate}
    
\section{Manfaat Penelitian}
    Penelitian ini dapat menambah wawasan penulis dan pembaca mengenai topik penelitian ini. Selain itu penelitian ini dapat menjadi referensi tambahan bagi peneliti berikutnya yang ingin membahas topik serupa.           % File: BAB I.tex (Berisi Pendahuluan)
\chapter{TINJAUAN KEPUSTAKAAN}
\section{Penggunaan Gambar Tunggal dan Lebih Dari Satu Agar Lebih Jelas}
    Ini contoh teks untuk mengisi template saja. Jadi ini bisa diabaikan karena tidak ada yang penting di dalamnya. Ini contoh teks untuk mengisi template saja. Jadi ini bisa diabaikan karena tidak ada yang penting di dalamnya. Ini contoh teks untuk mengisi template saja. Jadi ini bisa diabaikan karena tidak ada yang penting di dalamnya.
    \subsection{Gambar tunggal}
        Jika anda ingin memasukkan gambar tunggal, maka anda bisa menambahkannya seperti Gambar \ref{contoh-gambar} dan \ref{contoh-gambar2}. Ini contoh teks untuk mengisi template saja. Jadi ini bisa diabaikan karena tidak ada yang penting di dalamnya. Ini contoh teks untuk mengisi template saja. Jadi ini bisa diabaikan karena tidak ada yang penting di dalamnya. Ini contoh teks untuk mengisi template saja. Jadi ini bisa diabaikan karena tidak ada yang penting di dalamnya.
        
        \begin{figure}[!h]
        \centering
        \includegraphics[width=0.4\linewidth]{figures/contoh_figure.png}
        \caption{Profil solusi soliton persamaan SDNLS dengan parameter \(\beta = \qty{2.5}{}\); parameter \(\sigma\) bervariasi yaitu \(\sigma=\qty{2.5}{}; \qty{2.9}{}; \qty{3.3}{}; \qty{3.7}{}\); \(\Omega=1\); \(V_n = 0\); dan \(\alpha=0\) \citep{vinarky2026})}
        \label{contoh-gambar}
        \end{figure}
        
        Ini contoh teks untuk mengisi template saja. Jadi ini bisa diabaikan karena tidak ada yang penting di dalamnya. Ini contoh teks untuk mengisi template saja. Jadi ini bisa diabaikan karena tidak ada yang penting di dalamnya. Ini contoh teks untuk mengisi template saja. Jadi ini bisa diabaikan karena tidak ada yang penting di dalamnya. Ini contoh teks untuk mengisi template saja. Jadi ini bisa diabaikan karena tidak ada yang penting di dalamnya.
        
        \begin{figure}[!h]
        \centering
        \includegraphics[width=0.5\linewidth]{figures/contoh_figure.png}
        \caption{Profil solusi soliton persamaan SDNLS dengan parameter \(\beta = \qty{2.5}{}\); parameter \(\sigma\) bervariasi yaitu \(\sigma=\qty{2.5}{}; \qty{2.9}{}; \qty{3.3}{}; \qty{3.7}{}\); \(\Omega=1\); \(V_n = 0\); dan \(\alpha=0\) \citep{vinarky2026})}
        \label{contoh-gambar2}
        \end{figure}
        
        Ini contoh teks untuk mengisi template saja. Jadi ini bisa diabaikan karena tidak ada yang penting di dalamnya. Ini contoh teks untuk mengisi template saja. Jadi ini bisa diabaikan karena tidak ada yang penting di dalamnya. Ini contoh teks untuk mengisi template saja. Jadi ini bisa diabaikan karena tidak ada yang penting di dalamnya. Ini contoh teks untuk mengisi template saja. Jadi ini bisa diabaikan karena tidak ada yang penting di dalamnya. Ini contoh teks untuk mengisi template saja. Jadi ini bisa diabaikan karena tidak ada yang penting di dalamnya. Ini contoh teks untuk mengisi template saja. Jadi ini bisa diabaikan karena tidak ada yang penting di dalamnya. Ini contoh teks untuk mengisi template saja. Jadi ini bisa diabaikan karena tidak ada yang penting di dalamnya. Ini contoh teks untuk mengisi template saja. Jadi ini bisa diabaikan karena tidak ada yang penting di dalamnya.
        
    \subsection{Beberapa gambar}
        Jika anda ingin memasukkan beberapa gambar dalam satu gambar, maka anda bisa menambahkannya seperti Gambar \ref{beberapa-gambar}. Ini contoh teks untuk mengisi template saja. Jadi ini bisa diabaikan karena tidak ada yang penting di dalamnya. Ini contoh teks untuk mengisi template saja. Jadi ini bisa diabaikan karena tidak ada yang penting di dalamnya. Ini contoh teks untuk mengisi template saja. Ini contoh teks untuk mengisi template saja. Jadi ini bisa diabaikan karena tidak ada yang penting di dalamnya. Ini contoh teks untuk mengisi template saja. Jadi ini bisa diabaikan karena tidak ada yang penting di dalamnya. 
        
        \begin{figure}[!h]
            \centering
            \begin{subfigure}[b]{0.43\textwidth}
                \centering
                \includegraphics[width=\textwidth]{figures/contoh_subfigure1.png}
                \caption{}
            \end{subfigure}
            \begin{subfigure}[b]{0.43\textwidth}
                \centering
                \includegraphics[width=\textwidth]{figures/contoh_subfigure2.png}
                \caption{}
            \end{subfigure}
            \\
            \caption{Solusi penuh persamaan SDNLS dengan parameter \(\beta = \qty{2.5}{}\); parameter \(\sigma=\qty{2.5}{}\); \(\Omega=1\); \(V_n = 0\); serta (a) \(\alpha=0\); (b) \(\alpha=\qty{0.1}{}\)}
            \label{beberapa-gambar}
        \end{figure}
        
        Ini contoh teks untuk mengisi template saja. Jadi ini bisa diabaikan karena tidak ada yang penting di dalamnya. Ini contoh teks untuk mengisi template saja. Jadi ini bisa diabaikan karena tidak ada yang penting di dalamnya. Ini contoh teks untuk mengisi template saja. Jadi ini bisa diabaikan karena tidak ada yang penting di dalamnya. Ini contoh teks untuk mengisi template saja. Jadi ini bisa diabaikan karena tidak ada yang penting di dalamnya. Ini contoh teks untuk mengisi template saja. Jadi ini bisa diabaikan karena tidak ada yang penting di dalamnya. Ini contoh teks untuk mengisi template saja. Jadi ini bisa diabaikan karena tidak ada yang penting di dalamnya.

\section{Penggunaan Tabel}
Teori-teori dalam tinjauan kepustakaan mungkin saja membutuhkan tabel. Anda dapat membuat tabel seperti tabel \ref{contoh-tabel1} atau tabel \ref{contoh-tabel2}. Ini contoh teks untuk mengisi template saja. Jadi ini bisa diabaikan karena tidak ada yang penting di dalamnya. Ini contoh teks untuk mengisi template saja. Jadi ini bisa diabaikan karena tidak ada yang penting di dalamnya. Ini contoh teks untuk mengisi template saja. Jadi ini bisa diabaikan karena tidak ada yang penting di dalamnya.

    \begin{table}[!h]
        \centering
        \caption{Contoh tabel kita}
        \label{contoh-tabel1}
        \begin{tabular}{|c|l|r|} % | untuk garis vertikal, c=tengah, l=kiri, r=kanan
            \hline
            No. & Parameter & Hasil \\
            \hline
            1 & Kecepatan & 300 m/s \\
            2 & Frekuensi & 50 Hz \\
            3 & Suhu & 25$^\circ$C \\
            \hline
        \end{tabular}
    \end{table}
    
    Ini contoh teks untuk mengisi template saja. Jadi ini bisa diabaikan karena tidak ada yang penting di dalamnya. Ini contoh teks untuk mengisi template saja. Jadi ini bisa diabaikan karena tidak ada yang penting di dalamnya. Ini contoh teks untuk mengisi template saja. Jadi ini bisa diabaikan karena tidak ada yang penting di dalamnya. Ini contoh teks untuk mengisi template saja. Jadi ini bisa diabaikan karena tidak ada yang penting di dalamnya. Ini contoh teks untuk mengisi template saja. Jadi ini bisa diabaikan karena tidak ada yang penting di dalamnya. Ini contoh teks untuk mengisi template saja. Jadi ini bisa diabaikan karena tidak ada yang penting di dalamnya.
        
    \begin{table}[!h]
        \centering
        \caption{Tabel seperti ini sering kita temukan di berbagai artikel ilmiah karena bentuknya yang elegan dan sederhana tanpa ada garis-garis pemisah kolom}
        \label{contoh-tabel2}
        \begin{tabular}{ccc} % 'ccc' artinya 3 kolom, semuanya rata tengah (center)
            \toprule
            No. & Nama Variabel & Nilai \\
            \midrule
            1 & Indeks Bias ($n_1$) & 1.45 \\
            2 & Sudut Kritis ($\theta_c$) & 42\textdegree \\
            3 & Panjang Gelombang ($\lambda$) & 1550 nm \\
            \bottomrule
        \end{tabular}
    \end{table}
    
    Ini contoh teks untuk mengisi template saja. Jadi ini bisa diabaikan karena tidak ada yang penting di dalamnya. Ini contoh teks untuk mengisi template saja. Jadi ini bisa diabaikan karena tidak ada yang penting di dalamnya. Ini contoh teks untuk mengisi template saja. Jadi ini bisa diabaikan karena tidak ada yang penting di dalamnya. Ini contoh teks untuk mengisi template saja. Jadi ini bisa diabaikan karena tidak ada yang penting di dalamnya. Ini contoh teks untuk mengisi template saja.          % File: BAB II.tex (Berisi Tinjauan Pustaka)
\chapter{METODE PENELITIAN}
\section{Waktu dan Lokasi Penelitian}
    Anda dapat menjelaskan waktu dan lokasi penelitian di sini.
    
\section{Alat dan Bahan}
    Anda dapat mendeskripsikan alat dan bahan di sini.
    
\section{Cara Kerja}
    Anda dapat menjelaskan cara kerja di sini. Anda dapat membuat subbab lainnya yang dibutuhkan.         % File: BAB III.tex (Berisi Metode Penelitian)

% Bab IV dan V biasanya dinonaktifkan (%) saat tahap Proposal
%\chapter{HASIL DAN PEMBAHASAN}
\section{Hasil dan Pembahasan 1}
    Anda dapat menjelaskan hasil dan pembahasan dan penelitian anda di sini.
\section{Hasil dan Pembahasan 2}
    Anda dapat menjelaskan hasil dan pembahasan dan penelitian anda di sini.
\section{Dan Seterusnya}
    Anda dapat menambahkan subbab lainnya jika ada pembahasan lagi.
          % File: BAB IV.tex (Hasil dan Pembahasan)
%\chapter{KESIMPULAN DAN SARAN}
\section{Kesimpulan}
    Dari penelitian ini dapat disimpulkan beberapa hal sebagai berikut.

\section{Saran}
    Penulis menyarankan hal-hal berikut bagi peneliti selanjutnya.           % File: BAB V.tex (Kesimpulan dan Saran)

% --- Bagian Akhir (Back Matter) ---
\daftarKepustakaan{references} % Memanggil referensi dari file references.bib
\addcontentsline{toc}{chapter}{\vspace*{-1.5em}}
\addcontentsline{toc}{subsubsection}{\textbf{\hspace{1.5em}LAMPIRAN}}

\begin{lampiran}{Data Lengkap Penelitian}
    Anda dapat melampirkan apa yang ingin anda lampirkan di sini
\end{lampiran}

\begin{lampiran}{Listing Kode Program}
    Anda dapat melampirkan apa yang ingin anda lampirkan di sini
\end{lampiran}

\begin{lampiran}{Dokumentasi Penelitian}
    Anda dapat melampirkan apa yang ingin anda lampirkan di sini
\end{lampiran}               % Memasukkan file Lampiran.tex (jika ada)

\end{document}
% =============================================================================
