\chapter{TINJAUAN KEPUSTAKAAN}
\section{Penggunaan Gambar Tunggal dan Lebih Dari Satu Agar Lebih Jelas}
    Ini contoh teks untuk mengisi template saja. Jadi ini bisa diabaikan karena tidak ada yang penting di dalamnya. Ini contoh teks untuk mengisi template saja. Jadi ini bisa diabaikan karena tidak ada yang penting di dalamnya. Ini contoh teks untuk mengisi template saja. Jadi ini bisa diabaikan karena tidak ada yang penting di dalamnya.
    \subsection{Gambar tunggal}
        Jika anda ingin memasukkan gambar tunggal, maka anda bisa menambahkannya seperti Gambar \ref{contoh-gambar} dan \ref{contoh-gambar2}. Ini contoh teks untuk mengisi template saja. Jadi ini bisa diabaikan karena tidak ada yang penting di dalamnya. Ini contoh teks untuk mengisi template saja. Jadi ini bisa diabaikan karena tidak ada yang penting di dalamnya. Ini contoh teks untuk mengisi template saja. Jadi ini bisa diabaikan karena tidak ada yang penting di dalamnya.
        
        \begin{figure}[!h]
        \centering
        \includegraphics[width=0.4\linewidth]{figures/contoh_figure.png}
        \caption{Profil solusi soliton persamaan SDNLS dengan parameter \(\beta = \qty{2.5}{}\); parameter \(\sigma\) bervariasi yaitu \(\sigma=\qty{2.5}{}; \qty{2.9}{}; \qty{3.3}{}; \qty{3.7}{}\); \(\Omega=1\); \(V_n = 0\); dan \(\alpha=0\) \citep{vinarky2026})}
        \label{contoh-gambar}
        \end{figure}
        
        Ini contoh teks untuk mengisi template saja. Jadi ini bisa diabaikan karena tidak ada yang penting di dalamnya. Ini contoh teks untuk mengisi template saja. Jadi ini bisa diabaikan karena tidak ada yang penting di dalamnya. Ini contoh teks untuk mengisi template saja. Jadi ini bisa diabaikan karena tidak ada yang penting di dalamnya. Ini contoh teks untuk mengisi template saja. Jadi ini bisa diabaikan karena tidak ada yang penting di dalamnya.
        
        \begin{figure}[!h]
        \centering
        \includegraphics[width=0.5\linewidth]{figures/contoh_figure.png}
        \caption{Profil solusi soliton persamaan SDNLS dengan parameter \(\beta = \qty{2.5}{}\); parameter \(\sigma\) bervariasi yaitu \(\sigma=\qty{2.5}{}; \qty{2.9}{}; \qty{3.3}{}; \qty{3.7}{}\); \(\Omega=1\); \(V_n = 0\); dan \(\alpha=0\) \citep{vinarky2026})}
        \label{contoh-gambar2}
        \end{figure}
        
        Ini contoh teks untuk mengisi template saja. Jadi ini bisa diabaikan karena tidak ada yang penting di dalamnya. Ini contoh teks untuk mengisi template saja. Jadi ini bisa diabaikan karena tidak ada yang penting di dalamnya. Ini contoh teks untuk mengisi template saja. Jadi ini bisa diabaikan karena tidak ada yang penting di dalamnya. Ini contoh teks untuk mengisi template saja. Jadi ini bisa diabaikan karena tidak ada yang penting di dalamnya. Ini contoh teks untuk mengisi template saja. Jadi ini bisa diabaikan karena tidak ada yang penting di dalamnya. Ini contoh teks untuk mengisi template saja. Jadi ini bisa diabaikan karena tidak ada yang penting di dalamnya. Ini contoh teks untuk mengisi template saja. Jadi ini bisa diabaikan karena tidak ada yang penting di dalamnya. Ini contoh teks untuk mengisi template saja. Jadi ini bisa diabaikan karena tidak ada yang penting di dalamnya.
        
    \subsection{Beberapa gambar}
        Jika anda ingin memasukkan beberapa gambar dalam satu gambar, maka anda bisa menambahkannya seperti Gambar \ref{beberapa-gambar}. Ini contoh teks untuk mengisi template saja. Jadi ini bisa diabaikan karena tidak ada yang penting di dalamnya. Ini contoh teks untuk mengisi template saja. Jadi ini bisa diabaikan karena tidak ada yang penting di dalamnya. Ini contoh teks untuk mengisi template saja. Ini contoh teks untuk mengisi template saja. Jadi ini bisa diabaikan karena tidak ada yang penting di dalamnya. Ini contoh teks untuk mengisi template saja. Jadi ini bisa diabaikan karena tidak ada yang penting di dalamnya. 
        
        \begin{figure}[!h]
            \centering
            \begin{subfigure}[b]{0.43\textwidth}
                \centering
                \includegraphics[width=\textwidth]{figures/contoh_subfigure1.png}
                \caption{}
            \end{subfigure}
            \begin{subfigure}[b]{0.43\textwidth}
                \centering
                \includegraphics[width=\textwidth]{figures/contoh_subfigure2.png}
                \caption{}
            \end{subfigure}
            \\
            \caption{Solusi penuh persamaan SDNLS dengan parameter \(\beta = \qty{2.5}{}\); parameter \(\sigma=\qty{2.5}{}\); \(\Omega=1\); \(V_n = 0\); serta (a) \(\alpha=0\); (b) \(\alpha=\qty{0.1}{}\)}
            \label{beberapa-gambar}
        \end{figure}
        
        Ini contoh teks untuk mengisi template saja. Jadi ini bisa diabaikan karena tidak ada yang penting di dalamnya. Ini contoh teks untuk mengisi template saja. Jadi ini bisa diabaikan karena tidak ada yang penting di dalamnya. Ini contoh teks untuk mengisi template saja. Jadi ini bisa diabaikan karena tidak ada yang penting di dalamnya. Ini contoh teks untuk mengisi template saja. Jadi ini bisa diabaikan karena tidak ada yang penting di dalamnya. Ini contoh teks untuk mengisi template saja. Jadi ini bisa diabaikan karena tidak ada yang penting di dalamnya. Ini contoh teks untuk mengisi template saja. Jadi ini bisa diabaikan karena tidak ada yang penting di dalamnya.

\section{Penggunaan Tabel}
Teori-teori dalam tinjauan kepustakaan mungkin saja membutuhkan tabel. Anda dapat membuat tabel seperti tabel \ref{contoh-tabel1} atau tabel \ref{contoh-tabel2}. Ini contoh teks untuk mengisi template saja. Jadi ini bisa diabaikan karena tidak ada yang penting di dalamnya. Ini contoh teks untuk mengisi template saja. Jadi ini bisa diabaikan karena tidak ada yang penting di dalamnya. Ini contoh teks untuk mengisi template saja. Jadi ini bisa diabaikan karena tidak ada yang penting di dalamnya.

    \begin{table}[!h]
        \centering
        \caption{Contoh tabel kita}
        \label{contoh-tabel1}
        \begin{tabular}{|c|l|r|} % | untuk garis vertikal, c=tengah, l=kiri, r=kanan
            \hline
            No. & Parameter & Hasil \\
            \hline
            1 & Kecepatan & 300 m/s \\
            2 & Frekuensi & 50 Hz \\
            3 & Suhu & 25$^\circ$C \\
            \hline
        \end{tabular}
    \end{table}
    
    Ini contoh teks untuk mengisi template saja. Jadi ini bisa diabaikan karena tidak ada yang penting di dalamnya. Ini contoh teks untuk mengisi template saja. Jadi ini bisa diabaikan karena tidak ada yang penting di dalamnya. Ini contoh teks untuk mengisi template saja. Jadi ini bisa diabaikan karena tidak ada yang penting di dalamnya. Ini contoh teks untuk mengisi template saja. Jadi ini bisa diabaikan karena tidak ada yang penting di dalamnya. Ini contoh teks untuk mengisi template saja. Jadi ini bisa diabaikan karena tidak ada yang penting di dalamnya. Ini contoh teks untuk mengisi template saja. Jadi ini bisa diabaikan karena tidak ada yang penting di dalamnya.
        
    \begin{table}[!h]
        \centering
        \caption{Tabel seperti ini sering kita temukan di berbagai artikel ilmiah karena bentuknya yang elegan dan sederhana tanpa ada garis-garis pemisah kolom}
        \label{contoh-tabel2}
        \begin{tabular}{ccc} % 'ccc' artinya 3 kolom, semuanya rata tengah (center)
            \toprule
            No. & Nama Variabel & Nilai \\
            \midrule
            1 & Indeks Bias ($n_1$) & 1.45 \\
            2 & Sudut Kritis ($\theta_c$) & 42\textdegree \\
            3 & Panjang Gelombang ($\lambda$) & 1550 nm \\
            \bottomrule
        \end{tabular}
    \end{table}
    
    Ini contoh teks untuk mengisi template saja. Jadi ini bisa diabaikan karena tidak ada yang penting di dalamnya. Ini contoh teks untuk mengisi template saja. Jadi ini bisa diabaikan karena tidak ada yang penting di dalamnya. Ini contoh teks untuk mengisi template saja. Jadi ini bisa diabaikan karena tidak ada yang penting di dalamnya. Ini contoh teks untuk mengisi template saja. Jadi ini bisa diabaikan karena tidak ada yang penting di dalamnya. Ini contoh teks untuk mengisi template saja.